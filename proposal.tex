\documentclass{article}
\usepackage{graphicx}

\begin{document}

\title{Independent Study: Multiprocessor Weenix}
\author{Eric Caruso, Jackson Owens}
\maketitle

\section{Motivation}

Computers and even mobile devices are increasingly shipping with multiple cores. Any modern
operating system must be able to take advantage of all available processors. Because operating
systems are complicated, using ordinary locking schemes may lead to deadlock in the kernel which
is very hard to debug. Lock- and wait-free algorithms may alleviate this problem. In addition,
it might be tempting to use coarse-grained locking schemes because they are simple to implement,
but this cuts down the achievable parallelism.

\section{Overarching goals}

\begin{itemize}
    \item Learn and implement wait-free and lock-free algorithms and data structures
    \item Port Weenix to run on a multiprocessor
    \item Create a new project/tool for teaching OS design on multiprocessor systems
\end{itemize}

\section{Multiprocessor OS design}

For simplicity, Weenix is built on the assumption that:
\begin{enumerate}
    \item There is only one processor, and
    \item Whatever is running on that processor cannot be preempted.
\end{enumerate}

We wish to use this project to learn the complexities in designing and implementing
an operating system for a multiprocessor machine, because this is quite restrictive.

\section{Synchronization constructs}

We'll need to build up a library of synchronization constructs to use across the kernel. Many of
the data structures deeply embedded within Weenix, such as {\tt{list\_t}}, will need to be
replaced with concurrent versions.

\subsection{Hardware instruction abstractions}

To implement locks, wait-free and lock-free data structures, we'll need to use synchronization 
hardware instructions like x86's {\tt{CMPXCHG}}. To avoid littering inline assembly everywhere,
we'll create a library of wrappers around these synchronization instructions.

\subsection{Concurrent data structures}

Many kernel data structures like wait queues and lists must be accessible from multiple
processors concurrently. We'll need to write concurrent versions of these data structures.

\begin{itemize}
    \item \textbf{Mutexes}\\
    Some resources, like the hard disk, will need to be protected through mutual exclusion. The current mutex
    implementation, {\tt{kmutex\_t}}, will not work on a multiprocessor. We'll need to implement
    spin locks and mutexes.
    \item \textbf{Concurrent queue}\\
    Queues are used throughout Weenix, mostly in the form of {\tt{ktqueue\_t}} wait queues. We can
    use known algorithms for wait-free queues (Kogan, Petrank) and more practical lock-free queues (Michael, Scott) 
    \item \textbf{Concurrent list}\\
    Lists are everywhere in Weenix. Both lock-free and wait-free algorithms exist for linked lists.
    \item \textbf{Concurrent hash table or skip list}\\
    In Weenix page frames are looked up through a hash table. Algorithms exist for fast, concurrent hash tables. Or,
    we might be able to simplify things and instead use a concurrent skip list.
\end{itemize}

\section{Porting weenix}

Weenix is large enough that we should port incrementally. We can disable everything from kern 2 up, and follow the same
trajectory as 169, renabling parts of the operating system as we port.

\subsection{Subsystems and Kern 1}

The boot sequence must start the extra cores as appropriate. Slab allocators must be reworked to be SMP-safe,
and we would like to achieve this by using lock-free algorithms. The scheduler must also be changed. Since we
are implementing kernel preemption, we are going to need to use priority, and the scheduler will need to take
this into account as well. We will also need to change the page frame subsystem at this point, because
the page frames are needed to put the kernel at the proper location in virtual memory. TLB shootdown will also
need to be implemented.

\subsection{Kern 2 and Beyond}

Our implementations of the projects from here on usually made use of the single-processor, cooperative kernel
assumption, and will need more elaborate locking schemes, because there is now no guarantee that we can
execute contiguous statements without other things happening in between.

\section{Educational value}

cs167/9 currently teaches the design and implementation of operating systems, but as stated before this is
limited to systems which are not in use anymore. cs176 teaches the design of concurrent algorithms and
data structures, but in the class, the use cases are restricted to general software projects. As time goes
on, Weenix in its current form risks being too out-of-date to teach modern operating system design.

This project could be a remedy for this, and potentially take on several different forms. If the design
does not force students to jump many more complicated hurdles, it could be useful as a replacement for the
current Weenix project in 169. If the project does become too elaborate to be demonstrative of basic OS
design principles while allowing students to complete it in a semester, it could become a project in its
own right, either as a separate class or lab for people who have taken 167/9 and 176.

The case against replacing Weenix with this project is probably not due to extra workload, but the fact 
that it would become nondeterministic, which would make debugging more difficult for students.

\section{Backporting}

Over the course of developing SMP Weenix, we may find that changes can be ported back to the original
Weenix project if they make the design of the OS cleaner or the implementation easier.

\end{document}
